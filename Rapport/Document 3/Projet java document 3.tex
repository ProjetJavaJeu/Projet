	\documentclass[a4paper,titlepage]{article}
	\usepackage[utf8]{inputenc}
	\usepackage[export]{adjustbox}
	\usepackage{graphicx}
	\usepackage[margin=1.3in]{geometry}
	\author{Baptiste Vergote & Martin Schreinemachers}
	\begin{document}
	% Page de titre
	\titlepage{
		16 Décembre 2014 \\[2cm]
		\begin{center}\sf\Huge
		{\bfseries \underline{Document 3 :}} \\[2mm]
		{\bfseries Langage avancé de programmation} \\[1cm]
		\begin{figure}[!h]
		\centering
			\includegraphics[scale=0.55]{EcranCarte.png}
		\end{figure}
		\begin{center}
		{\huge RPG - The Epic School Adventure}
		
		\end{center}
		
		\end{center}
		\ \\[4cm]
		\textbf{Baptiste Vergote \& Martin Schreinemachers 2TL2} \\
		EPHEC LLN -- 2014-2015
	}
	\clearpage
	\tableofcontents
	\clearpage
	\section*{TODO}
	\begin{enumerate}
		\item Un document word comprenant : 
		\begin{enumerate}
			\item Une page de garde (noms, prénoms, class, titre du projet, un screenshot du projet, année académique).
			\item Une introduction où vous présentez votre projet.
			\item Le diagramme UML des classes OO que vous avez (ou un extrait intéressant).
			\item Un texte expliquant tout ce qui est nécessaire pour que votre projet fonctionne (modification du path, driver à installer et dans ce cas les fournir, …).
			\item Quelques screenshots de votre projet (idéalement pour illustrer des passages de votre travail – donc vous ne faites pas un titre screenshot mais vous en mettez là où c’est utile).
			\item Le code source (ou un extrait représentatif) documenté du package principal de l’application (+fichier package-info.java).
			\item La documentation du package principal (fichiers html générés par la javadoc).
			\item Une description de votre stratégie de validation.
			\item Une conclusion : que vous a apporté ce projet, quelles ont été les difficultés rencontrées, comment peut-on l’améliorer.
		\end{enumerate}
		\item Tous vos codes sources et tests unitaires (le projet eclipse).
		\item Le fichier JAR de votre application.
	\end{enumerate}
	\clearpage
	\section{Introduction}
	\textit{Une introduction où vous présentez votre projet. -> A SUPPRIMER}\\
	
	Dans un petite bourgade en plein centre du monde que j'attire votre attention...\\
	Dans les environs de Louvain-La-Neuve, c'est en amont d'un lac puant que l'aventure a démarré...\\
	
	Une journée bien trop remplie... 3 heures à rester assis, à écouter et suivre, sans un mot ( ou ?de cours d'affilées), un challenge que peu arrive à endurer. C'est en êtres surhumains que les deux apprentis programmeurs, dont nous parlerons plus tard, se sont lancés dans un projet fou, insensé et sans fin..\\
	
	Actuellement, ils sommes assis dans une classe trop sombre, trop petite, qui sent la transpiration... Interdit/empécher(autre à trouver?) d'aérer à cause des autochtones qui peuplent l'établissement... Ceux ci ne cesse de l'empester de leurs parfums et de leurs fumées ragoutantes/répugnantes... Quel supplice de devoir les supporter au quotidien...\\
	Ces derniers sont prèts à tout pour bouter nos héros hors de leur forteresse(chateau/royaume(?)).\\
	
	Mais nos deux héros, accompagnés d'une poignée, sont des durs à cuire, ils subissent des attaques de toute part mais ne flanchent pas(tiennent bon? (non, suite<-))..\\ 
	Ils tiennent tête(s?) à ces soldats fous, fervants défenseurs des grands champions, "Ceux-dont-on-ne-doit-pas-prononcer-le-nom".\\
	
	Encourage par leurs proches, nos deux programmeurs aliénés/déséquilibrés(=Fou a l'extreme) se sont lancés dans l'aventure/expédition/odyssée/épreuve.\\
	Accompagné de leur PC et de leurs cerveaus, affaiblis par les attaques répétées, ils ont commencé à imaginer un moyen d'arriver(non, autre) à leur but ultime (NO SPOIL !!).\\
	
	Serez vous assez fort pour les aider?\\
	Parviendrez vous à arrêter toutes les menaces auxquels vous serez confrontés?\\
	Si oui, suivez nous, arpentez vous dans ces contrées dangereuse et devenez protecteur des faibles et des opprimés!
	
	\clearpage
	\section{Diagramme UML}
	Le diagramme UML des classes OO que vous avez (ou un extrait intéressant).\\
	\clearpage
	
	\section{Necessité pour la fonctionnalité}
	Un texte expliquant tout ce qui est nécessaire pour que votre projet fonctionne (modification du path, driver à installer et dans ce cas les fournir, …).
	\clearpage
	
	\section{Captures d'écran}
	Quelques screenshots de votre projet (idéalement pour illustrer des passages de votre travail – donc vous ne faites pas un titre screenshot mais vous en mettez là où c’est utile).
	\subsection{Démarrage}
	C'est dans les contrées du bois de Lauzelle
	\begin{figure}[h!]
		\includegraphics[scale=0.7]{EcranDemarrage.png}
	\end{figure}
	
	\subsection{Création de personnage}
	Que tout à commencé	
	\begin{figure}[h!]
		\includegraphics[scale=0.7]{EcranCreationPersonnage.png}
	\end{figure}
	
	\subsection{La carte du jeu}
	\begin{figure}[h!]
		\includegraphics[scale=0.7]{EcranCarte.png}
	\end{figure}
	

	\subsection{Les combats}
	\begin{figure}[h!]
		\includegraphics[scale=0.7]{EcranCarte.png}
	\end{figure}	
	\clearpage
	
	\section{Documentation du package}
	La documentation du package principal (fichiers html générés par la javadoc).
	\clearpage
	
	\section{Description de la stratégie de validation}
	Une description de votre stratégie de validation.
	\clearpage
	
	\section{Conclusion}
	Une conclusion : que vous a apporté ce projet, quelles ont été les difficultés rencontrées, comment peut-on l’améliorer.
	\clearpage
	
	
	\section{Différentes classes}
		Il existe deux classes dans notre programme : le Guerrier et le Mage.\\
		Ces deux classes ont un attribut principal, et un attribut secondaire.\\
		Le niveau maximum du jeu est 10.
		
	\subsection{Guerrier}
		La force est la caractéristique principale du guerrier, celle ci augmente de 1 chaque level !\\
		L'endurance augmente aussi chaque level de 1\\
		Endurance initial à 1, *10 pour obtenir ses pv, force à 1 !\\
		Pour ce qui est de l'intelligence, elle est aussi élevée que celle de martin...\\ Je lui ai appliqué des valeurs (dans un tableau) en fonction du niveau du personnage : 	\{0,0,0,0,1,1,1,1,2,2\}
		
	\subsection{Mage}
		endurance initial à 1, *7 pour obtenir ses pv, force à 0, inteligence à 1 !
	\end{document}