\documentclass[a4paper,titlepage]{article}
\usepackage[utf8]{inputenc}
\author{Baptiste Vergote & Martin Schreinemachers}
\begin{document}
% Page de titre
\titlepage{
	\today \\[5cm]
	\begin{flushright}\sf\Huge
	{\bfseries \underline{Document 1 :}} \\[2mm]
	{\bfseries Langage avancé de programmation} \\[3mm]
	{\huge Mme Vroman \& \\Mme Van Den Schrieck}
	\end{flushright}
	\ \\[8cm]
	\textbf{Baptiste Vergote \& Martin Schreinemachers 2TL2} \\
	EPHEC LLN -- 2014-2015
}
\clearpage
\noindent
\tableofcontents
\clearpage
\section{TODO}
\begin{enumerate}
	\item Un document word comprenant : 
	\begin{enumerate}
		\item Une page de garde (noms, prénoms, class, titre du projet, un screenshot du projet, année académique).
		\item Une introduction où vous présentez votre projet.
		\item Le diagramme UML des classes OO que vous avez (ou un extrait intéressant).
		\item Un texte expliquant tout ce qui est nécessaire pour que votre projet fonctionne (modification du path, driver à installer et dans ce cas les fournir, …).
		\item Quelques screenshots de votre projet (idéalement pour illustrer des passages de votre travail – donc vous ne faites pas un titre screenshot mais vous en mettez là où c’est utile).
		\item Le code source (ou un extrait représentatif) documenté du package principal de l’application (+fichier package-info.java).
		\item La documentation du package principal (fichiers html générés par la javadoc)
		\item Une description de votre stratégie de validation.
		\item Une conclusion : que vous a apporté ce projet, quelles ont été les difficultés rencontrées, comment peut-on l’améliorer.
	\end{enumerate}
	\item Tous vos codes sources et tests unitaires (le projet eclipse).
	\item Le fichier JAR de votre application.
\end{enumerate}
\clearpage
\section{Introduction}
Une introduction où vous présentez votre projet. -> A SUPPRIMER\\
C'est dans les environs de Louvain-La-Neuve que l'aventure a démarré...\\
Nous sommes assis dans une classe trop sombre, trop petite et qui sent la transpiration

\clearpage
\section{Diagramme UML}
Le diagramme UML des classes OO que vous avez (ou un extrait intéressant).\\
\clearpage

\section{Necessité pour la fonctionnalité}
Un texte expliquant tout ce qui est nécessaire pour que votre projet fonctionne (modification du path, driver à installer et dans ce cas les fournir, …).
\clearpage

\section{Captures d'écran}
Quelques screenshots de votre projet (idéalement pour illustrer des passages de votre travail – donc vous ne faites pas un titre screenshot mais vous en mettez là où c’est utile).
\clearpage

\section{Documentation du package}
\clearpage

\section{Description de la stratégie de validation}
\clearpage

\section{Conclusion}
\clearpage


\section{Différentes classes}
	Il existe deux classes dans notre programme : le Guerrier et le Mage.\\
	Ces deux classes ont un attribut principal, et un attribut secondaire.\\
	Le niveau maximum du jeu est 10.
	
\subsection{Guerrier}
	La force est la caractéristique principale du guerrier, celle ci augmente de 1 chaque level !\\
	L'endurance augmente aussi chaque level de 1\\
	Endurance initial à 1, *10 pour obtenir ses pv, force à 1 !\\
	Pour ce qui est de l'intelligence, elle est aussi élevée que celle de martin...\\ Je lui ai appliqué des valeurs (dans un tableau) en fonction du niveau du personnage : 	\{0,0,0,0,1,1,1,1,2,2\}
	
\subsection{Mage}
	endurance initial à 1, *7 pour obtenir ses pv, force à 0, inteligence à 1 !
\end{document}