	\documentclass[a4paper,titlepage]{article}
	\usepackage[utf8]{inputenc}
	\usepackage[export]{adjustbox}
	\usepackage[french]{babel}
	\usepackage{hyperref}
	\usepackage{graphicx}
	\usepackage[margin=1.3in]{geometry}
	\author{Baptiste Vergote & Martin Schreinemachers}
	\begin{document}
	% Page de titre
	\titlepage{
		16 Décembre 2014 \\[2cm]
		\begin{center}\sf\Huge
		{\bfseries \underline{Document 3 :}} \\[2mm]
		{\bfseries Langage avancé de programmation} \\[1cm]
		\begin{figure}[!h]
		\centering
			\includegraphics[scale=0.55]{EcranCarte.png}
		\end{figure}
		\begin{center}
		{\huge RPG - The Epic School Adventure}
		
		\end{center}
		
		\end{center}
		\ \\[4cm]
		\textbf{Baptiste Vergote \& Martin Schreinemachers 2TL2} \\
		EPHEC LLN -- 2014-2015
	}
	\clearpage
	\tableofcontents
	\clearpage
	\section*{TODO}
	\begin{enumerate}
		\item Un document word comprenant : 
		\begin{enumerate}
			\item Une page de garde (noms, prénoms, class, titre du projet, un screenshot du projet, année académique).
			\item Une introduction où vous présentez votre projet.
			\item Le diagramme UML des classes OO que vous avez (ou un extrait intéressant).
			\item Un texte expliquant tout ce qui est nécessaire pour que votre projet fonctionne (modification du path, driver à installer et dans ce cas les fournir, …).
			\item Quelques screenshots de votre projet (idéalement pour illustrer des passages de votre travail – donc vous ne faites pas un titre screenshot mais vous en mettez là où c’est utile).
			\item Le code source (ou un extrait représentatif) documenté du package principal de l’application (+fichier package-info.java).
			\item La documentation du package principal (fichiers html générés par la javadoc).
			\item Une description de votre stratégie de validation.
			\item Une conclusion : que vous a apporté ce projet, quelles ont été les difficultés rencontrées, comment peut-on l’améliorer.
		\end{enumerate}
		\item Tous vos codes sources et tests unitaires (le projet eclipse).
		\item Le fichier JAR de votre application.
	\end{enumerate}
	\clearpage
	\section{Introduction}
	\textit{Une introduction où vous présentez votre projet. -> A SUPPRIMER}\\
	
	Dans un petite bourgade en plein centre du monde que j'attire votre attention...\\
	Dans les environs de Louvain-La-Neuve, c'est en amont d'un lac puant que l'aventure a démarré...\\
	
	Une journée bien trop remplie... 3 heures à rester assis, à écouter et suivre, sans un mot ( ou ?de cours d'affilées), un challenge que peu arrive à endurer. C'est en êtres surhumains que les deux apprentis programmeurs, dont nous parlerons plus tard, se sont lancés dans un projet fou, insensé et sans fin..\\
	
	Actuellement, ils sommes assis dans une classe trop sombre, trop petite, qui sent la transpiration... Interdit/empécher(autre à trouver?) d'aérer à cause des autochtones qui peuplent l'établissement... Ceux ci ne cesse de l'empester de leurs parfums et de leurs fumées ragoutantes/répugnantes... Quel supplice de devoir les supporter au quotidien...\\
	Ces derniers sont prèts à tout pour bouter nos héros hors de leur forteresse(chateau/royaume(?)).\\
	
	Mais nos deux héros, accompagnés d'une poignée, sont des durs à cuire, ils subissent des attaques de toute part mais ne flanchent pas(tiennent bon? (non, suite<-))..\\ 
	Ils tiennent tête(s?) à ces soldats fous, fervants défenseurs des grands champions, "Ceux-dont-on-ne-doit-pas-prononcer-le-nom".\\
	
	Encourage par leurs proches, nos deux programmeurs aliénés/déséquilibrés(=Fou a l'extreme) se sont lancés dans l'aventure/expédition/odyssée/épreuve.\\
	Accompagné de leur PC et de leurs cerveaus, affaiblis par les attaques répétées, ils ont commencé à imaginer un moyen d'arriver(non, autre) à leur but ultime (NO SPOIL !!).\\
	
	Serez vous assez fort pour les aider?\\
	
	Parviendrez vous à arrêter toutes les menaces auxquels vous serez confrontés?\\
	Si oui, suivez nous, arpentez vous dans ces contrées dangereuse et devenez protecteur des faibles et des opprimés!
	
	\clearpage
	\section{Diagramme UML}
	Le diagramme UML des classes OO que vous avez (ou un extrait intéressant).\\
	\clearpage
	
	\section{Necessité pour la fonctionnalité}
	Un texte expliquant tout ce qui est nécessaire pour que votre projet fonctionne (modification du path, driver à installer et dans ce cas les fournir, …).
	\clearpage
	
	\section{Notre jeu étapes par étapes}
	En double cliquant sur le fichier jar, le jeu se lance en plein écran, vous laissant apercevoir notre magnifique écran de démarrage :
	\begin{figure}[h!]
		\includegraphics[scale=0.30]{EcranDebut.jpg}
	\end{figure}
	\\Sur cette image, trois boutons sont présents : 
	\begin{enumerate}
		\item Le bouton de création d'une Nouvelle Partie.
		\item La bouton permettant de Continuer la partie en cours.
		\item Un bouton pour changer les options.
	\end{enumerate}
	Dans cet exemple, nous partirons du principe que vous n'avez pas encore eu l'occasion de tester le jeu ! Nous exécuterons donc la création d'une Nouvelle Partie.\\
	\\En cliquant sur cette case, un écran de sélection de personnage apparait :
	\begin{figure}[h!]
		\includegraphics[scale=0.30]{EcranCreationPersonnage.jpg}
	\end{figure}
	\\Vous cliquez sur le personnage que vous souhaitez jouer, lui entrez un Pseudo et le jeu commence :
	\begin{figure}[h!]
		\includegraphics[scale=0.65]{EcranCarte.png}
	\end{figure}
	\\Comme vous pouvez le constater, le guerrier apparait en plein milieu de la carte, celle ci délimitée par les petits cailloux que vous ne pourriez, bien entendu, pas traverser ;) !\\
	\\En se baladant dans la prairie, celui ci va se heurter à des rencontres inattendues ou notre champion va devoir vendre chèrement sa peau !\\
	\\Nous ne dévoilerons pas la suite, à vous de la découvrir ! Bonne chance dans cet univers rempli de monstres sanguinaires !
	\clearpage
	
	\section{Documentation du package}
	La documentation du package principal (fichiers html générés par la javadoc).
	\clearpage
	
	\section{Description de la stratégie de validation}
	Une description de votre stratégie de validation.
	\clearpage
	
	\section{Difficultés rencontrées}
	\begin{itemize}
		\item Avoir un jeu fonctionnel !\\
		==$>$ Beaucoup de petits problèmes sont survenus
		\item L'interface Slick2D complexe et longue à maîtriser
	\end{itemize}
	\section{Conclusion}
	Une conclusion : que vous a apporté ce projet, quelles ont été les difficultés rencontrées, comment peut-on l’améliorer.\\
	\\
	\\
	Martin adorait l'idée d'un RPG, étant tous les deux des joueurs de RPG, nous nous sommes lancés très vite dans l'idée.
	\clearpage
	\section*{Sources }
	Pour faire une biblio correcte :\\	\url{http://www.uclouvain.be/cps/ucl/doc/bpsp/documents/Bibliographie\_APA\_F\_13doi.pdf}

	\subsection*{Page Web}
Dans la bibliographie, la référence d’une page Web
(site Internet) comprend
quatre zones, séparées chacune par un point :
 \begin{itemize}
  \item La zone auteur : deux cas peuvent se présenter :
  \begin{itemize}
  	\item La page Web est signée par un auteur : il faut alors inscrire son nom de famille en minuscule à l’exception de la première lettre, suivi d’une virgule, suivie des initiales des prénoms en majuscules, chaque initiale étant suivie d’un point. Si il y a plusieurs auteurs, chaque auteur sera séparé par un ; un \& sera ajouté entre l’avant-dernier et le dernier auteur.
  	\item Il n’y a pas d’auteur : il faut alors indiquer l’organisme responsable du site en toutes lettres. 
  \end{itemize}
  \item La zone année de publication : il faut reprendre la date figurant sur le document virtuel entre parenthèses. Si il n’y a pas de date, il faut la remplacer par \og s.d.\fg{}. 
  \item La zone titre : à part les noms propres et les acronymes, seule la première lettre du titre et du sous-titre de la page Web seront en majuscules. \textit{Cette zone sera mise en évidence soit en la mettant en italique, soit en la soulignant.}
  \item La zone de localisation Internet : cette zone est introduite par la mention \og En Ligne \fg{}  suivie de l’adresse URL. Si la page Web est susceptible de changer dans le temps, celle-ci sera suivie d’une virgule, suivie de la mention \og consulté le \fg{}, suivie de la date de consultation sous la forme jour/mois/année. \textbf{Attention} : si l’auteur a été clairement identifié, l’adresse URL sera complétée par \og En ligne sur le site Web de \fg{} suivi du nom de l’organisme responsable du site. \textbf{Attention} : si le site, dans la zone du copyright, indique un DOI, c’est celui-ci qui doit être indiqué en lieu et place de la zone de localisation Internet. 
 \end{itemize}
 
\subsubsection*{Exemple:}
	American Psychological Association. (2009).\textit{Controlling anger before it controls you.} En ligne \url{http://www.apa.org/topics/anger/control.aspx}.\\
	
	Feyereisen, P. (2002). \textit{Le vieillissement cognitif.} En ligne sur le site de l’Université
catholique de Louvain, Unité de cognition et développement \url{http://www.code.ucl.ac.be/vico.html}.\\

	Wikipédia. (2010).\textit{Emotion}. En ligne \url{http://fr.wikipedia.org/wiki/Emotion}, consulté le 6 mai 2010. 
	
	
	\clearpage	
	\section*{Différentes classes}
		Il existe deux classes dans notre programme : le Guerrier et le Mage.\\
		Ces deux classes ont un attribut principal, et un attribut secondaire.\\
		Le niveau maximum du jeu est 10.
		
	\subsection*{Guerrier}
		La force est la caractéristique principale du guerrier, celle ci augmente de 1 chaque level !\\
		L'endurance augmente aussi chaque level de 1\\
		Endurance initial à 1, *10 pour obtenir ses pv, force à 1 !\\
		Pour ce qui est de l'intelligence, elle est aussi élevée que celle de martin...\\ Je lui ai appliqué des valeurs (dans un tableau) en fonction du niveau du personnage : 	\{0,0,0,0,1,1,1,1,2,2\}
		
	\subsection*{Mage}
		endurance initial à 1, *7 pour obtenir ses pv, force à 0, inteligence à 1 !
	\end{document}